\begin{frame}{Week 10 — UML Diagram}
\small
\noindent

  \centering
  \centering\includegraphics[height=0.4\linewidth]{image/UML.jpg}

  \vspace{20pt}
  \rule{\textwidth}{0.5pt}
  \vspace{20pt}
  
  \raggedright
  
  \textbf{How I found the visualization (and why I picked it)}\\
  I found this visualization while browsing through my software engineering textbook. I chose it because it represents the structure of a software system using UML (Unified Modeling Language), which is a fundamental concept in software design. I picked it because I have seen this kind of visualization in many software engineering courses during my studies.


  \vfill % Espace vertical flexible
  
  \textbf{What this visualization shows}\\
  The chart displays a UML class diagram, where each box represents a class in the software system. The lines connecting the boxes indicate relationships between the classes, such as inheritance, association, or dependency. The layout allows users to quickly identify the main components of the software system and understand how they interact with each other.
  \vfill % Espace vertical flexible
  
  \textbf{Context}\\
  This visualization is commonly used in software engineering to analyze and understand the structure of software systems. It is meant to provide insights into the design and architecture of the system, helping developers and engineers make informed decisions based on the relationships between different components. The design is tailored to the needs of professionals working with software systems, providing a clear and concise representation of class structures and interactions.
  \vfill % Espace vertical flexible
  
  \textbf{Design choices}\\
  The use of boxes to represent classes and lines to represent relationships allows for an intuitive visualization of software structure, while the layout supports analysis of the overall architecture of the system. However, the complexity of the diagram can make it difficult to read for those unfamiliar with UML notation.
  \vfill % Espace vertical flexible
  
  \textbf{Improvements}\\
  The diagram could be improved by adding contextual elements, such as annotations or color coding, runtime information, or interactive features that allow users to explore the relationships between classes in more detail.
  \vspace{0pt} % Alignement en bas

\end{frame}
