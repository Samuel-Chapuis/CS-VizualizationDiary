\begin{frame}{Week 11 — Snow Prediction}
\small
\noindent
\begin{minipage}{0.4\textwidth}
  \centering
  \rotatebox{90}{\centering\includegraphics[height=0.8\linewidth]{image/snow.png}}
  \hspace{20pt}
  \vrule width 0.5pt%
\end{minipage}
\begin{minipage}{0.5\textwidth}
  \centering
  \setlength{\parskip}{3mm}%
  \justifying
  \vspace{0pt} % Alignement en haut
  
  \textbf{How I found the visualization (and why I picked it)}\\
  I encountered this visualization while reading scientific literature related to phase-change phenomena in cold environments (snow and ice modeling).
  I selected it because it is not a conventional chart (e.g. time series or map), but a space–time diagram, a representation frequently used in physics and geosciences yet rarely discussed from a visualization-design perspective.
  \vfill % Espace vertical flexible
  
  \textbf{What this visualization shows}\\
  The visualization shows the temporal evolution of the ice layer profile at several horizontal positions $x/H$.
  \begin{itemize}
    \item The horizontal axis represents time.
    \item The vertical axis represents ice layer thickness or vertical position.
    \item Color encodes the state or magnitude of the ice layer (e.g.\ thickness, presence/absence of ice).
  \end{itemize}
  The diagram allows the reader to observe:
  \begin{itemize}
    \item Growth and decay phases of the ice layer,
    \item Temporal synchronization or delay between different horizontal positions,
    \item The presence of moving fronts or phase-transition boundaries.
  \end{itemize}
  \vfill % Espace vertical flexible
  
  \textbf{Context}\\
  This type of visualization is typically used in cryospheric physics, thermodynamics, and phase-change modeling, where understanding the coupling between spatial heterogeneity and temporal evolution is crucial.
  In the context of ice formation and melting:
    \begin{itemize}
      \item Local conditions vary with horizontal position,
      \item Phase transitions occur over time rather than instantaneously,
      \item Interfaces between ice and water (or air) are key physical objects.
    \end{itemize}
  The space–time diagram is therefore used as an analytical tool rather than a purely descriptive one, aiming to reveal dynamic patterns rather than precise numerical values.
  
  \textbf{Design choices}\\
  Several deliberate design choices are visible:
  \begin{itemize}
    \item \textbf{Space–time encoding:} Time is treated as a geometric dimension rather than a sequence of frames, enabling pattern recognition across long durations.
    \item \textbf{Color mapping:} Color is used to encode the magnitude or state of the ice layer, allowing phase boundaries to appear as continuous structures.
    \item \textbf{Multiple panels:} Separate subplots for different $x/H$ values reduce overlap and allow comparison between spatial locations.
    \item \textbf{Continuous representation:} The absence of explicit gridlines emphasizes continuity but reduces precision.
  \end{itemize}
  While these choices support qualitative understanding, they prioritize pattern recognition over exact reading.
  \vfill % Espace vertical flexible
  
  \textbf{Improvements}\\
  Several improvements could enhance both perceptual clarity and analytical usefulness such as, explicit color scale and units, phase boundary highlighting, uncertainty representation or temporal landmarks.
  \vspace{0pt} % Alignement en bas
\end{minipage}
\end{frame}
