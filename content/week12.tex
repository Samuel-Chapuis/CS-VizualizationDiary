\begin{frame}{Week 12 — Pressure Enthalpy Diagram}
\small
\noindent

  \centering
  \centering\includegraphics[height=0.4\linewidth]{image/water-p-h-diagram_orig.jpg}

  \vspace{20pt}
  \rule{\textwidth}{0.5pt}
  \vspace{20pt}
  
  \raggedright
  
  \textbf{How I found the visualization (and why I picked it)}\\
  This visualization is a pressure-enthalpy (P-h) diagram for water and come from my thermodynamics class. I chose it because it is probably one of the most complex and information-dense visualizations I have ever seen. 

  \vfill % Espace vertical flexible
  
  \textbf{What this visualization shows}\\
  The diagram displays the thermodynamic properties of water, with pressure on the y-axis and enthalpy on the x-axis. The various lines and curves represent different phases (solid, liquid, gas) and processes (isothermal, isobaric, isentropic) that water can undergo. The layout allows users to analyze the behavior of water under different conditions and to understand the relationships between pressure, enthalpy, and phase changes.
  \vfill % Espace vertical flexible
  
  \textbf{Context}\\
  This type of diagram is commonly used in thermo-hydrolic to analyse the phase of the water in a system. It could be for example used to understand the behavior of water in a power plant, where water is heated to produce steam that drives turbines. 
  \vfill % Espace vertical flexible
  
  \textbf{Design choices}\\
  The use of a pressure-enthalpy diagram allows for a comprehensive visualization of the thermodynamic properties of water, while the layout supports analysis of phase changes and processes. However, the complexity of the diagram can make it difficult to read for those unfamiliar with thermodynamics, and the dense information may require careful study to fully understand.
  \vfill % Espace vertical flexible
  
  \textbf{Improvements}\\
  For me having a computational interface would actually be a great improvement for this visualization. Because this for me shouldn't be a static diagram but an interactive tool where you can input different parameters (e.g., temperature, pressure) and see how the diagram changes in real-time. 
  \vspace{0pt} % Alignement en bas

\end{frame}
