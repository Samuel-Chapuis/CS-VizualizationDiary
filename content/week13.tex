\begin{frame}{Week 13 — Emission Spectrum}
\small
\noindent
\begin{minipage}{0.4\textwidth}
  \centering
  \rotatebox{90}{\centering\includegraphics[height=0.8\linewidth]{image/emission_spectrum.png}}
  \hspace{20pt}
  \vrule width 0.5pt%
\end{minipage}
\begin{minipage}{0.5\textwidth}
  \centering
  \setlength{\parskip}{3mm}%
  \justifying
  \vspace{0pt} % Alignement en haut
  
  \textbf{How I found the visualization (and why I picked it)}\\
  This visualization is are multiple line plots showing wavelengths of light absorbed by different gases. I've been using this to see how the laser we were using during my internship was absorbed by the different gases in the air. I chose it because it is a very clear and concise way to visualize the absorption spectrum of different gases, which is a fundamental concept in spectroscopy and atmospheric science.
  \vfill % Espace vertical flexible
  
  \textbf{What this visualization shows}\\
  The chart displays the absorption spectrum of different gases, where the x-axis represents the wavelength of light and the black lines represent the absorption features of different gases. Each line corresponds to a specific gas, and the depth of the line indicates the strength of absorption at that wavelength.
  \vfill % Espace vertical flexible
  
  \textbf{Context}\\
  This visualization is commonly used in spectroscopy and atmospheric science to analyze the absorption characteristics of different gases. 
  
  \textbf{Design choices}\\
  The use of the color and the layout allows for an intuitive visualization of the absorption spectrum, while the line plot format supports analysis of the absorption features of different gases.
  \vfill % Espace vertical flexible
  
  \textbf{Improvements}\\
  To improve this visualization i would extend it in the infrared range to see the absorption of the different gases in this range as well. I would also add a legend to indicate which line corresponds to which electron transition and maybe add a second plot showing the emission spectrum of the same gases for comparison.
  \vspace{0pt} % Alignement en bas
\end{minipage}
\end{frame}
