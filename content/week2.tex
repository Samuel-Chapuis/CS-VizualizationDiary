\begin{frame}{Week 2 — Sinistres Visualization}
\small
\noindent

  \centering
  \centering\includegraphics[height=0.4\linewidth]{image/sinistres.png}

  \vspace{20pt}
  \rule{\textwidth}{0.5pt}
  \vspace{20pt}
  
  \raggedright
  
  \textbf{How I found the visualization (and why I picked it)}\\
  My dad made this visualization for his work at the insurance company. I chose it because it represents a real-world application of data visualization in a professional context. It immediately caught my attention because it condenses a large amount of information about insurance claims into a single, visually appealing chart.

  \vfill % Espace vertical flexible
  
  \textbf{What this visualization shows}\\
  The chart displays the cost per block of insured building in multiple sinistres (claims) for the year 2023. Each block represents a building, and the height of the block corresponds to the cost of the claim. The layout allows users to quickly identify which buildings had higher claims and to compare costs across different blocks.

  \vfill % Espace vertical flexible
  
  \textbf{Context}\\
  This visualization is used by the insurance company to analyze the distribution of claims and to identify area where the claims can be regrouped. It is meant to help the company understand the risk associated with different buildings and to make informed decisions about insurance policies and premiums. The design is tailored to the needs of insurance professionals, providing a clear overview of claim costs.

  \vfill % Espace vertical flexible
  
  \textbf{Design choices}\\
  The use of vertical blocks allows for an intuitive comparison of claim costs, while the map background provides geographical context.
  \vfill % Espace vertical flexible
  
  \textbf{Improvements}\\
  For a more analytical audience, I would add interactive tooltips showing exact claim costs and building details. A second line chart could reveal trends in claim costs over time. Finally, alternative color schemes could be used to differentiate between types of claims or to highlight areas with higher risks.
  
  \vspace{0pt} % Alignement en bas

\end{frame}
