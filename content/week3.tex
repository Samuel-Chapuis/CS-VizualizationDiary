\begin{frame}{Week 3 — Stocks Visualization}
\small
\noindent
\begin{minipage}{0.4\textwidth}
  \centering
  \rotatebox{90}{\centering\includegraphics[height=0.8\linewidth]{image/stocks.png}}
  \hspace{20pt}
  \vrule width 0.5pt%
\end{minipage}
\begin{minipage}{0.5\textwidth}
  \centering
  \setlength{\parskip}{3mm}%
  \justifying
  \vspace{0pt} % Alignement en haut
  
  \textbf{How I found the visualization (and why I picked it)}\\
  I encountered this visualization directly on my bank account dashboard. I chose it because it represents the trace of one of my investments in the stock market. It immediately attracted my attention because it condenses an entire year of stock performance into a single chart.
  \vfill % Espace vertical flexible
  
  \textbf{What this visualization shows}\\
  The chart displays the performance of a stock over the last two months using a candlestick visualization. The x-axis represents time, while the y-axis represents the stock price. Each candlestick shows the opening, closing, high, and low prices for a given period, with the body indicating the open-close range and wicks showing the price extremes. This format allows users to see trends, volatility, and key price movements at a glance. The layout follows a standard financial chart format, making it familiar to users who track investments.

  \vfill % Espace vertical flexible
  
  \textbf{Context}\\
  This visualization is automatically generated by the bank to summarize the performance of the stock in which I have invested. It is meant to provide a quick overview of how the stock has performed over time, helping investors make informed decisions about buying, holding, or selling their shares. The design is tailored to the needs of investors, providing a clear and concise representation of stock performance.

  \vfill % Espace vertical flexible
  
  \textbf{Design choices}\\
  The use of candlesticks allows for a detailed view of price movements, while the time-series layout supports trend analysis. However, the chart could benefit from additional annotations to highlight significant events (e.g., earnings reports, market news) that may have influenced stock performance. The color scheme typically uses green for upward movements and red for downward movements, which is intuitive but may not be accessible to color-blind users.

  \vfill % Espace vertical flexible
  
  \textbf{Improvements}\\
  For a more analytical audience, I would add interactive tooltips showing exact price values and dates. A second line chart could reveal moving averages or volume trends. Finally, alternative color schemes with higher contrast or color-blind-friendly options (e.g., blue–orange) would improve accessibility without losing the financial context.

  \vspace{0pt} % Alignement en bas
\end{minipage}
\end{frame}
