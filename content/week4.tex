\begin{frame}{Week 4 — Streamline over Wing}
\small
\noindent

  \centering
  \centering\includegraphics[height=0.4\linewidth]{image/streamlines.png}

  \vspace{20pt}
  \rule{\textwidth}{0.5pt}
  \vspace{20pt}
  
  \raggedright
  
  \textbf{How I found the visualization (and why I picked it)}\\
  As a Aerospace engineering, I have been exposed to various visualizations related to aerodynamics and fluid dynamics. I chose this particular visualization because it represents the flow of air over a wing, which is a selfexplanatory and fundamental concept in aerospace engineering. It immediately caught my attention because it condenses complex aerodynamic phenomena into a single, visually appealing chart that illustrates the behavior of airflow around a wing.

  \vfill % Espace vertical flexible
  
  \textbf{What this visualization shows}\\
  The chart displays the streamline patterns of airflow over a wing. Each line represents the path that air particles take as they flow around the wing, with the density and curvature of the lines indicating the speed and direction of the airflow. The layout allows users to quickly identify areas of high and low pressure, as well as regions of turbulence and separation.

  \vfill % Espace vertical flexible
  
  \textbf{Context}\\
  This visualization is common output of computational fluid dynamics (CFD) simulations used in aerospace engineering to analyze and optimize the aerodynamic performance of wings and other aircraft components. It is meant to help engineers understand the complex interactions between airflow and wing geometry, enabling them to make informed design decisions that improve lift, reduce drag, and enhance overall aircraft performance.

  \vfill % Espace vertical flexible
  
  \textbf{Design choices}\\
  The use of streamline patterns allows for an intuitive visualization of airflow behavior, while the wing shape provides a clear reference for the aerodynamic context.
  \vfill % Espace vertical flexible
  
  \textbf{Improvements}\\
  For a more analytical audience, I would add interactive tooltips showing exact airflow velocity and pressure values at different points. A second visualization could reveal pressure distribution or turbulence intensity. Finally, alternative color schemes could be used to highlight different flow characteristics or to improve accessibility.
  
  \vspace{0pt} % Alignement en bas

\end{frame}
