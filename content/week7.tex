\begin{frame}{Week 7 — Energy Flow Diagram}
\small
\noindent
\begin{minipage}{0.4\textwidth}
  \centering
  \rotatebox{90}{\centering\includegraphics[height=0.8\linewidth]{image/energy_flow_chart.png}}
  \hspace{20pt}
  \vrule width 0.5pt%
\end{minipage}
\begin{minipage}{0.5\textwidth}
  \centering
  \setlength{\parskip}{3mm}%
  \justifying
  \vspace{0pt} % Alignement en haut
  
  \textbf{How I found the visualization (and why I picked it)}\\
  I encountered this visualization as an exemple of a Matlab project during my last internship. I chose it because it represents the flow of energy in a system, which is a fundamental concept in physics and engineering. Even if its simplicity, it immediately caught my attention because it condenses complex energy interactions into a single, visually appealing chart that illustrates the behavior of energy flow in a system.

  \vfill % Espace vertical flexible
  
  \textbf{What this visualization shows}\\
  The chart displays the energy flow in a system, where each arrow represents the flow of energy from one component to another. The thickness of the arrows indicates the magnitude of the energy flow, while the color may represent different types of energy (e.g., kinetic, potential, thermal). The layout allows users to quickly identify the main sources and sinks of energy, as well as the overall structure of energy interactions within the system.

  \vfill % Espace vertical flexible
  
  \textbf{Context}\\
  This visualization is commonly used in physics and engineering to analyze and understand energy interactions in various systems, such as mechanical systems, electrical circuits, or thermodynamic processes. It is meant to provide insights into the behavior of energy flow, helping researchers and engineers make informed decisions based on the energy dynamics of the system. The design is tailored to the needs of professionals working with energy systems, providing a clear and concise representation of energy flow.

  \vfill % Espace vertical flexible
  
  \textbf{Design choices}\\
  The use of arrows to represent energy flow allows for an intuitive visualization of energy interactions, while the layout supports analysis of the overall structure of energy flow. 
  \vfill % Espace vertical flexible
  
  \textbf{Improvements}\\
  For me there is not much to improve in this visualization it is already very clear and concise.

  \vspace{0pt} % Alignement en bas
\end{minipage}
\end{frame}
