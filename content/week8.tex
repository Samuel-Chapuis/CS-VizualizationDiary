\begin{frame}{Week 8 — Pressure Gadiet Visualization}
\small
\noindent

  \centering
  \centering\includegraphics[height=0.4\linewidth]{image/gradiant.png}

  \vspace{20pt}
  \rule{\textwidth}{0.5pt}
  \vspace{20pt}
  
  \raggedright
  
  \textbf{How I found the visualization (and why I picked it)}\\
  I found this visualization while browsing through my physics textbook. I chose it because it represents the evolution of the turbulence over a sphere, which is a fundamental concept in fluid dynamics. I picked it because I have seen this kind of visualization in many scientific classes during my aerospace engineering studies.


  \vfill % Espace vertical flexible
  
  \textbf{What this visualization shows}\\
  The chart displays the pressure gradient around a sphere, where each color represents a different pressure level. The layout allows users to quickly identify areas of high and low pressure, as well as the overall structure of the pressure distribution around the sphere.
  \vfill % Espace vertical flexible
  
  \textbf{Context}\\
  This visualization is commonly used in fluid dynamics to analyze and understand the behavior of fluids around objects, such as spheres, cylinders, or airfoils. It is meant to provide information about the pressure distribution, which is crucial for understanding phenomena like lift, drag, and turbulence. 
  \vfill % Espace vertical flexible
  
  \textbf{Design choices}\\
  The use of a color map to represent pressure levels allows for an intuitive visualization of pressure distribution, while the layout supports analysis of the overall structure of pressure around the sphere. However, the choice of color map can significantly impact the readability. Here the gradiant goes from blue to red which is a common choice for pressure visualizations, but it needs to be familiar with this color scheme.
  \vfill % Espace vertical flexible

  \textbf{Improvements}\\
  A video demonstrating the pressure gradient evolution over time around the sphere would provide better understanding of how pressure distribution changes dynamically. This would be especially useful for analyzing transient phenomena and fluid behavior in different flow conditions.
  
  \vspace{0pt} % Alignement en bas

\end{frame}
