\begin{frame}{Week 9 — Pie Pie Chart}
\small
\noindent
\begin{minipage}{0.4\textwidth}
  \centering
  \rotatebox{0}{\centering\includegraphics[height=0.8\linewidth]{image/pie_chart.png}}
  \hspace{20pt}
  \vrule width 0.5pt%
\end{minipage}
\begin{minipage}{0.5\textwidth}
  \centering
  \setlength{\parskip}{3mm}%
  \justifying
  \vspace{0pt} % Alignement en haut
  
  \textbf{How I found the visualization (and why I picked it)}\\
  This is obviously a joke visualization that I found on the internet. I chose it because it represents a humorous take on the traditional pie chart, which is a common type of data visualization. It immediately attracted my attention because it subverts the expectations of what a pie chart should look like, using actual pies to represent data segments in a playful and visually engaging way.

  \vfill % Espace vertical flexible
  
  \textbf{What this visualization shows}\\
  The chart displays a pie chart made out of actual pies, where each slice of the pie represents a different category of pie filling (e.g., apple, cherry, pumpkin). The size of each slice corresponds to the proportion of each pie filling in the overall chart. 
  \vfill % Espace vertical flexible
  
  \textbf{Context}\\
  This visualization is meant to be a humorous and creative representation of data, rather than a serious analytical tool. It is designed to entertain and engage viewers by using a familiar format (the pie chart) in an unexpected way (with real pies). 
  
  \textbf{Design choices}\\
  The use of actual pies to represent the segments of the pie chart adds a playful and visually engaging element to the visualization. The layout mimics a traditional pie chart, making it easy to understand while adding a humorous twist.
  \vfill % Espace vertical flexible
  
  \textbf{Improvements}\\
  As this is a joke visualization, there is not much to improve in terms of design or functionality.

  \vspace{0pt} % Alignement en bas
\end{minipage}
\end{frame}
